\documentclass[]{article}
\usepackage{lmodern}
\usepackage{amssymb,amsmath}
\usepackage{ifxetex,ifluatex}
\usepackage{fixltx2e} % provides \textsubscript
\ifnum 0\ifxetex 1\fi\ifluatex 1\fi=0 % if pdftex
  \usepackage[T1]{fontenc}
  \usepackage[utf8]{inputenc}
\else % if luatex or xelatex
  \ifxetex
    \usepackage{mathspec}
  \else
    \usepackage{fontspec}
  \fi
  \defaultfontfeatures{Ligatures=TeX,Scale=MatchLowercase}
\fi
% use upquote if available, for straight quotes in verbatim environments
\IfFileExists{upquote.sty}{\usepackage{upquote}}{}
% use microtype if available
\IfFileExists{microtype.sty}{%
\usepackage{microtype}
\UseMicrotypeSet[protrusion]{basicmath} % disable protrusion for tt fonts
}{}
\usepackage[margin=1in]{geometry}
\usepackage{hyperref}
\hypersetup{unicode=true,
            pdftitle={Introduction to the use of fisflix},
            pdfauthor={Nina Schiettekatte},
            pdfborder={0 0 0},
            breaklinks=true}
\urlstyle{same}  % don't use monospace font for urls
\usepackage{natbib}
\bibliographystyle{plainnat}
\usepackage{color}
\usepackage{fancyvrb}
\newcommand{\VerbBar}{|}
\newcommand{\VERB}{\Verb[commandchars=\\\{\}]}
\DefineVerbatimEnvironment{Highlighting}{Verbatim}{commandchars=\\\{\}}
% Add ',fontsize=\small' for more characters per line
\usepackage{framed}
\definecolor{shadecolor}{RGB}{248,248,248}
\newenvironment{Shaded}{\begin{snugshade}}{\end{snugshade}}
\newcommand{\AlertTok}[1]{\textcolor[rgb]{0.94,0.16,0.16}{#1}}
\newcommand{\AnnotationTok}[1]{\textcolor[rgb]{0.56,0.35,0.01}{\textbf{\textit{#1}}}}
\newcommand{\AttributeTok}[1]{\textcolor[rgb]{0.77,0.63,0.00}{#1}}
\newcommand{\BaseNTok}[1]{\textcolor[rgb]{0.00,0.00,0.81}{#1}}
\newcommand{\BuiltInTok}[1]{#1}
\newcommand{\CharTok}[1]{\textcolor[rgb]{0.31,0.60,0.02}{#1}}
\newcommand{\CommentTok}[1]{\textcolor[rgb]{0.56,0.35,0.01}{\textit{#1}}}
\newcommand{\CommentVarTok}[1]{\textcolor[rgb]{0.56,0.35,0.01}{\textbf{\textit{#1}}}}
\newcommand{\ConstantTok}[1]{\textcolor[rgb]{0.00,0.00,0.00}{#1}}
\newcommand{\ControlFlowTok}[1]{\textcolor[rgb]{0.13,0.29,0.53}{\textbf{#1}}}
\newcommand{\DataTypeTok}[1]{\textcolor[rgb]{0.13,0.29,0.53}{#1}}
\newcommand{\DecValTok}[1]{\textcolor[rgb]{0.00,0.00,0.81}{#1}}
\newcommand{\DocumentationTok}[1]{\textcolor[rgb]{0.56,0.35,0.01}{\textbf{\textit{#1}}}}
\newcommand{\ErrorTok}[1]{\textcolor[rgb]{0.64,0.00,0.00}{\textbf{#1}}}
\newcommand{\ExtensionTok}[1]{#1}
\newcommand{\FloatTok}[1]{\textcolor[rgb]{0.00,0.00,0.81}{#1}}
\newcommand{\FunctionTok}[1]{\textcolor[rgb]{0.00,0.00,0.00}{#1}}
\newcommand{\ImportTok}[1]{#1}
\newcommand{\InformationTok}[1]{\textcolor[rgb]{0.56,0.35,0.01}{\textbf{\textit{#1}}}}
\newcommand{\KeywordTok}[1]{\textcolor[rgb]{0.13,0.29,0.53}{\textbf{#1}}}
\newcommand{\NormalTok}[1]{#1}
\newcommand{\OperatorTok}[1]{\textcolor[rgb]{0.81,0.36,0.00}{\textbf{#1}}}
\newcommand{\OtherTok}[1]{\textcolor[rgb]{0.56,0.35,0.01}{#1}}
\newcommand{\PreprocessorTok}[1]{\textcolor[rgb]{0.56,0.35,0.01}{\textit{#1}}}
\newcommand{\RegionMarkerTok}[1]{#1}
\newcommand{\SpecialCharTok}[1]{\textcolor[rgb]{0.00,0.00,0.00}{#1}}
\newcommand{\SpecialStringTok}[1]{\textcolor[rgb]{0.31,0.60,0.02}{#1}}
\newcommand{\StringTok}[1]{\textcolor[rgb]{0.31,0.60,0.02}{#1}}
\newcommand{\VariableTok}[1]{\textcolor[rgb]{0.00,0.00,0.00}{#1}}
\newcommand{\VerbatimStringTok}[1]{\textcolor[rgb]{0.31,0.60,0.02}{#1}}
\newcommand{\WarningTok}[1]{\textcolor[rgb]{0.56,0.35,0.01}{\textbf{\textit{#1}}}}
\usepackage{longtable,booktabs}
\usepackage{graphicx,grffile}
\makeatletter
\def\maxwidth{\ifdim\Gin@nat@width>\linewidth\linewidth\else\Gin@nat@width\fi}
\def\maxheight{\ifdim\Gin@nat@height>\textheight\textheight\else\Gin@nat@height\fi}
\makeatother
% Scale images if necessary, so that they will not overflow the page
% margins by default, and it is still possible to overwrite the defaults
% using explicit options in \includegraphics[width, height, ...]{}
\setkeys{Gin}{width=\maxwidth,height=\maxheight,keepaspectratio}
\IfFileExists{parskip.sty}{%
\usepackage{parskip}
}{% else
\setlength{\parindent}{0pt}
\setlength{\parskip}{6pt plus 2pt minus 1pt}
}
\setlength{\emergencystretch}{3em}  % prevent overfull lines
\providecommand{\tightlist}{%
  \setlength{\itemsep}{0pt}\setlength{\parskip}{0pt}}
\setcounter{secnumdepth}{0}
% Redefines (sub)paragraphs to behave more like sections
\ifx\paragraph\undefined\else
\let\oldparagraph\paragraph
\renewcommand{\paragraph}[1]{\oldparagraph{#1}\mbox{}}
\fi
\ifx\subparagraph\undefined\else
\let\oldsubparagraph\subparagraph
\renewcommand{\subparagraph}[1]{\oldsubparagraph{#1}\mbox{}}
\fi

%%% Use protect on footnotes to avoid problems with footnotes in titles
\let\rmarkdownfootnote\footnote%
\def\footnote{\protect\rmarkdownfootnote}

%%% Change title format to be more compact
\usepackage{titling}

% Create subtitle command for use in maketitle
\newcommand{\subtitle}[1]{
  \posttitle{
    \begin{center}\large#1\end{center}
    }
}

\setlength{\droptitle}{-2em}

  \title{Introduction to the use of fisflix}
    \pretitle{\vspace{\droptitle}\centering\huge}
  \posttitle{\par}
    \author{Nina Schiettekatte}
    \preauthor{\centering\large\emph}
  \postauthor{\par}
    \date{}
    \predate{}\postdate{}
  
\usepackage{graphicx}
\usepackage{float}

\begin{document}
\maketitle

\hypertarget{introduction}{%
\subsection{Introduction}\label{introduction}}

The \texttt{fishflux} package provides a tool to model fluxes of C
(carbon), N (nitrogen) and P (phosphorus) in fish. It combines basic
priciples from elemental stoichiometry and metabolic theory. The package
offers a userfriendly interface to make nutrient dynamic modelling
available for anyone. \texttt{fishflux} is mostly targeted towards fish
ecologists, wishing to predict nutrient ingestion, egestion and
excretion to study fluxes of nutrients and energy.

Main assets:

\begin{itemize}
\tightlist
\item
  Provides functions to model fluxes of Carbon, Nitrogen and Phosphorus
  for fish with or without the MCMC sampler provided by stan.
\item
  Provides some tools to find the right parameters as inputs into the
  model
\item
  Provides a plotting function to illustrate results
\end{itemize}

\hypertarget{installing-and-loading-fishflux}{%
\subsection{Installing and loading
fishflux}\label{installing-and-loading-fishflux}}

\texttt{fishflux} uses Markov Chain Monte Carlo simulations provided by
\href{https://github.com/stan-dev/rstan/wiki/RStan-Getting-Started}{stan}.
Therefore, the first step is to install
\href{https://github.com/stan-dev/rstan/wiki/RStan-Getting-Started}{stan}.

\hypertarget{github}{%
\subsubsection{GitHub}\label{github}}

The best way to install the latest development version of
\texttt{fishflux} is to instal it from GitHub.

\begin{Shaded}
\begin{Highlighting}[]
\KeywordTok{install.packages}\NormalTok{(}\StringTok{"devtools"}\NormalTok{)}
\NormalTok{devtools}\OperatorTok{::}\KeywordTok{install_github}\NormalTok{(}\StringTok{"nschiett/fishflux"}\NormalTok{, }\DataTypeTok{dependencies=}\OtherTok{TRUE}\NormalTok{)}
\KeywordTok{library}\NormalTok{(fishflux)}
\end{Highlighting}
\end{Shaded}

\hypertarget{cran}{%
\subsubsection{CRAN}\label{cran}}

\texttt{fishflux} will be available on CRAN in the future:

\begin{Shaded}
\begin{Highlighting}[]
\KeywordTok{install.packages}\NormalTok{(}\StringTok{"fishflux"}\NormalTok{)}
\KeywordTok{library}\NormalTok{(fishflux)}
\end{Highlighting}
\end{Shaded}

\hypertarget{downloaded-package-file}{%
\subsubsection{Downloaded package file}\label{downloaded-package-file}}

Another option is to download the source file available on github
\href{https://github.com/nschiett/fishflux}{here}.

\begin{Shaded}
\begin{Highlighting}[]
\KeywordTok{install.packages}\NormalTok{(path_to_fishflux_file, }\DataTypeTok{repos =} \OtherTok{NULL}\NormalTok{, }\DataTypeTok{type=}\StringTok{"source"}\NormalTok{)}
\KeywordTok{library}\NormalTok{(fishflux)}
\end{Highlighting}
\end{Shaded}

\hypertarget{how-to-use-fishflux}{%
\subsubsection{How to use fishflux?}\label{how-to-use-fishflux}}

\texttt{fishflux} is designed to follow three simple steps: 1. Find the
right input parameters 2. Run the model simulation with those input
parameters 3. Plot the model results

\hypertarget{find-parameters}{%
\paragraph{Find parameters}\label{find-parameters}}

Below a table showing all parameters needed to run the mosel simulation.
\texttt{fishflux} provides functions to find some of these parameters,
but note that others have to be provided by the user at this stage.

\begin{longtable}[]{@{}ll@{}}
\caption{Species specific parameters. \(L_{inf}\) and k are parameters
of the Von Beffalanty growth curve. Fp, Fn and Fc are the element
contents of the fish's diet. k is a parameter defining the activity
level of the fish}\tabularnewline
\toprule
species & \(L_{inf}\)\tabularnewline
\midrule
\endfirsthead
\toprule
species & \(L_{inf}\)\tabularnewline
\midrule
\endhead
a & expla\tabularnewline
b & explb\tabularnewline
\bottomrule
\end{longtable}

A good place to start is to check if you are using the correct
scientific name of your fish of interest. The function
\texttt{name\_errors} will tell you if the species name is correct. This
function can be useful, especially when working with larger databases.

\begin{Shaded}
\begin{Highlighting}[]
\NormalTok{fishflux}\OperatorTok{::}\KeywordTok{name_errors}\NormalTok{(}\StringTok{"Zebrazoma scopas"}\NormalTok{)}
\end{Highlighting}
\end{Shaded}

\begin{verbatim}
## Inaccurate species names found:
\end{verbatim}

\begin{verbatim}
## [1] "Zebrazoma scopas"
\end{verbatim}

When we have verified or corrected the species name we can start finding
some parameters.

The \texttt{find\_lw} function searches fishbase to find length-weight
relationship parameters \texttt{lw\_a} and \texttt{lw\_b} obtained from
\emph{Froese, R., J. Thorson and R.B. Reyes Jr., 2013. A Bayesian
approach for estimating length-weight relationships in fishes. J. Appl.
Ichthyol. (2013):1-7.}

\begin{Shaded}
\begin{Highlighting}[]
\NormalTok{fishflux}\OperatorTok{::}\KeywordTok{find_lw}\NormalTok{(}\StringTok{"Zebrasoma scopas"}\NormalTok{)}
\end{Highlighting}
\end{Shaded}

\begin{verbatim}
##            species   lwa_m     lwa_sd lwb_m    lwb_sd
## 1 Zebrasoma scopas 0.02455 0.00272449  2.98 0.0255102
\end{verbatim}

\texttt{growth\_params()}, \texttt{trophic\_level()} and ``


\end{document}
